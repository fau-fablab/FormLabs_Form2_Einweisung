% !TeX spellcheck = de_DE_frami
%%%%%%%%%%%%%%%%%%%%%%%%%%%%%%%%%%%%%%%%%%%%%%%%
% COPYRIGHT: (C) 2012-2015 FAU FabLab and others
% Bearbeitungen ab 2015-02-20 fallen unter CC-BY-SA 3.0
% Sobald alle Mitautoren zugestimmt haben, steht die komplette Datei unter CC-BY-SA 3.0. Bis dahin ist der Lizenzstatus aller alten Bestandteile ungeklärt.
%%%%%%%%%%%%%%%%%%%%%%%%%%%%%%%%%%%%%%%%%%%%%%%%


\newcommand{\basedir}{fablab-document}
\documentclass{\basedir/fablab-document}

\usepackage{amssymb} % Symbole für Knöpfe
\usepackage{subfigure,caption}
\usepackage{eurosym}
\usepackage{textcomp} % \textcelsius
\usepackage{tabularx} % Tabellen mit bestimmtem Breitenverhältnis der Spalten
\usepackage{wrapfig} % Textumlauf um Bilder
\usepackage{float} % Ermöglicht H als Platzierungsoption
\renewcommand{\texteuro}{\euro}
\newcommand{\fachbegriff}[1]{(\textit{#1})}
\newcommand{\ts}[1]{\textsuperscript{#1}}
\newcommand{\ra}{$\Rightarrow$}

% \linespread{1.2}
% \fancyhead{}
\date{\today}
\author{kontakt@fablab.fau.de}
\title{Einweisung SLA 3D-Drucker}

\begin{document}
	
	\maketitle
	\begin{center}
		Für den 3D-Drucker \textbf{FormLabs Form2}
	\end{center}
	
	\textbf{Nur eingewiesene Benutzer dürfen den Drucker selbstständig benutzen}, um teure Beschädigungen zu vermeiden. Wenn du noch nicht eingewiesen bist, frage einen Betreuer. Er erklärt dir die Bedienung und lässt dich unter Aufsicht dein gewünschtes Teil ausdrucken. Wenn du alles verstanden hast, darfst auch du dann die Einweisung unterschreiben und den Drucker in Zukunft selbstständig verwenden.
	
	\section{Regeln und Hinweise}
	% Hier sollte nur das wichtigste gegen "Kaputtmachen" des Druckers stehen.
	Für die Benutzung ist es wichtig, dass du folgende Hinweise beachtest:
	
	\begin{itemize}
		\item Obwohl das 3D-Drucken unter den Begriff ``Rapid Prototyping'' fällt, kann ein Druck je nach Größe und
		Präzision gut mehrere Stunden bis Tage dauern. Betreuer als auch die druckerspezifische Software helfen dir, die Dauer abzuschätzen.
		\item Nicht unbeaufsichtigt drucken, immer wieder mal einen Blick darauf werfen, besonders am Anfang.\\
		Wenn du nicht bis zum Ende deines Drucks da sein kannst, frage vorher einen Betreuer und hinterlasse einen Zettel mit Name und Kontaktdaten.
		\item Anleitung exakt beachten. Wenn du nicht weiter weißt oder dir unsicher bist, frag einen Betreuer.
		\item Drucker nicht ausschalten,
		\item Materialwechsel und sonstige Wartung darf nur gemeinsam mit einem Betreuer durchgeführt werden.
		\item \textbf{Wiegen und Bezahlen nicht vergessen!}
	\end{itemize}
	\newpage
	
	% % % % % % % % % % % % % % % %
	
	\renewcommand{\contentsname}{Inhaltsverzeichnis / Arbeitsablauf}
	\setcounter{tocdepth}{2}
	\tableofcontents
	\newpage
	
	% % % % % % % % % % % % % % % %
	\section{3D-Modell erstellen}
	\subsection{Dateiformat}
	
	Im STL-Dateiformat, Einheit: Millimeter. Alle gängigen 3D-Programme haben einen STL-Export.
	
	\begin{itemize}
		\item auf \href{https://thingiverse.com}{Thingiverse.com} gibt es viele vorgefertigte Modelle, als
		Grundlage oder gleich zum fertig ausdrucken.
		\item oder erstelle es mit einem Programm deiner Wahl
		\begin{table}[H]
			\centering
			\begin{tabularx}{\textwidth}{|l|X|}
				\hline \textbf{Name} & \textbf{Beschreibung} \\
				\hline \multicolumn{2}{|c|}{\textit{kostenlose Software}}  \\
				\hline Blender & relativ komplex aber auch für Freiformflächen geeignet  \\
				\hline OpenSCAD & Skriptsprache für Konstruktion aus geometrischen Grundkörpern \\
				\hline DesignSpark Mechanical & Angelehnt an professionelle CAD-Software, aber relativ einfach zu bedienen  \\
				\hline TinkerCAD & sehr einfach, für Kinder gut geeignet  \\
				\hline Google SketchUp & wenig Einarbeitung, geringer Funktionsumfang, für einfache Teile \\
				\hline & \\
				\hline \multicolumn{2}{|c|}{\textit{kostenpflichtige Software (proprietär)}}  \\
				\hline PTC Creo, Solid Edge, Siemens NX & kostenlos beim RRZE für Studenten, professionelle Software \\
				\hline Autocad Inventor & kostenlos bei Autodesk für Studenten ebenfalls für professionelle Anwendungen \\
				\hline
			\end{tabularx}
		\end{table}
		\item Die 3D-Daten müssen gewisse Regeln erfüllen. Bei Modellierungsprogrammen wie Blender ist etwas Vorsicht oder Nacharbeit nötig, die meisten Konstruktionsprogramme (Solid Edge und Konsorten, auch OpenSCAD) machen es prinzipbedingt von selber richtig. Zu den Einschränkungen bei Blender stehen am Ende der Anleitung noch weitere Informationen.
	\end{itemize}
	
	\subsection{Einschränkungen der Formen}
	\begin{itemize}
			\item \textbf{Maximale Abmessungen}  L:145 B:145 H:175mm
			\item zur Sicherheit lieber ein paar Millimeter kleiner.
			\item Es ist in der Praxis meist nicht möglich, den Bauraum des Druckers auch nur annähernd auszunutzen! Wenn möglich, gestalte deine Objekte kleiner als 10x10x5cm (LxBxH).
		\item \textbf{große Teile dauern ewig}, während des Drucks muss jemand dabeibleiben. Druckzeit für 30x30x30mm sind je
		nach Präzision etwa 30-45 Minuten, ein größeres Volumen braucht entsprechend länger.
		\item Durch das Druckverfahren sind gewisse Formen schlecht druckbar. Oft ist Ausprobieren angesagt.
		\item Damit man auch \enquote{schwierige} Formen drucken kann, kann die Software \textbf{Stützstrukturen} erstellen. Wenn man dies erlaubt,
		erzeugt der Drucker ein loses Geflecht unter Überhängen und Brücken,
		das sich nach dem Ausdrucken mit einer Zange oder einem Skalpell entfernen lässt. So kann
		man diese Begrenzungen umgehen. Nachteil ist die schlechtere
		Oberflächenqualität und der Nachbereitungs-Aufwand.
	\end{itemize}
	\newpage
	
	% % % % % % % % % % % %
	
	\section{Vorbereitung}
	
	\subsection{Vorheizen}
	Vorheizen ist nicht nötig, da es automatisch vor dem Druck geschieht.
	
	\subsection{Material auswählen - PLA vs. ABS}
	Grundsätzlich lässt sich PLA etwas einfacher und besser verarbeiten als ABS. Für ABS benötigt man z.\,B. eine beheiztes
	Bett, das im Ultimaker Original nicht vorhanden ist. Mit PLA tut sich der Replicator dafür schwerer. Für die meisten
	Anwendungen ist es egal welches Material man verwendet, die Entscheidung richtet sich mehr nach der persönlichen
	Vorliebe. Im Internet lässt sich einiges zu diesem Thema finden.
	
	\subsubsection{PLA}
	\begin{itemize}
		\item Organisches Material, Biokunststoff
		\item weniger elastisch, wird aber bei 70\textcelsius{} weich
		\item benötigt nicht zwingend ein beheiztes Bett (wenn vorhanden Bett trotzdem auf ca. 90\textcelsius{} aufheizen)
		\item Drucktemperatur beträgt meistens 210\textcelsius{}. Je nach Material kann sie allerdings variieren.
	\end{itemize}

	\subsection{Materialwechsel}
	Wir haben verschiedene Harze auf Lager. Wenn du eine andere möchtest, frag einfach nach.
	Der Materialwechsel sollte aber \textbf{nur durch einen Betreuer} erfolgen. (Infos für Betreuer $\to$ \ref{filamentwechsel})
	
	% % % % % % % % % % % %
	
	\section{3D-Modell mit PreForm umwandeln und ausdrucken}

	\begin{itemize}
		\item Programm PreForm öffnen und Drucker "PrettyTiger" auswählen. und mit "Auswählen bestätigen.
		\item mit Klick auf den "Datei"\,Dialog und anschließend "öffnen"\,STL-Datei öffnen
		\item bei Bedarf das Modell mit den Schaltflächen links skalieren, drehen und verschieben.
		\item notwendige Stützstrukturen automatisch erstellen lassen, bei Bedarf händisch anpassen.
		\item der Drucker ist im Netzwerk und wird automatisch von der Software erkannt
		\item jeder Druck (auch Fehldrucke oder Abbrüche) ist zu wiegen und abzurechnen
		\item Harztank nicht mit Deckel im Drucker lagern, bei Stromanschluss führt der Drucker einen Selbsttest durch und bewegt den Wischer. Alternative: Sicherstellen dass der Drucker nicht angeschaltet werden kann (z.Bsp. Stecker ziehen)
		\end{itemize}
	
	\section{Bezahlen und abschließen}
	
	Um das Objekt von der Platte zu lösen vorsichtig arbeiten. Meistens lässt es sich von Hand lösen. Wenn nicht,
	warten bis sich die Platte etwas abgekühlt hat. \textbf{Nicht versuchen, das Objekt mit scharfen oder spitzen Gegenständen herunter zu hebeln!}
	Sollte das Tape oder die Folie auf der Plattform beim Herunterlösen kaputt gehen, bitte einen \textbf{Betreuer es zu erneuern}.
	
	Drucker reinigen, siehe \ref{putzen}.
	
	Objekt mit Feinwaage (steht meist bei den 3D-Druckern) abwiegen, Preis pro Gramm ist im Kassensystem eingetragen.
	
	Es muss alles mitgewogen werden, auch die Stützstruktur und der Müll, den der Extruder anfangs ausspuckt. \textbf{Fehldrucke müssen ebenfalls bezahlt werden.}
	\pagebreak
	
	% % % % % % % % % % % %
	
	\section{Zusatzinfos für Benutzer}
	
	\subsection{Manuelles Verfahren}\label{manuelles-verfahren}
	Manchmal muss man vor einem Druck, z.\,B. um die Plattform oder die Düse zu säubern, den Extruder bewegen.
	Das darf bei beiden Druckern \textbf{nicht von Hand} gemacht werden!
	
	\subsubsection{Ultimaker Original}
	\begin{itemize}
		\item Am Drucker selbst aus dem Hauptmenü ''Prepare`` \ra ''Move Axis``
		\item hier können die Achsen bewegt werden
	\end{itemize}
	Beim \textbf{Ultimaker\ts2} können die Achsen nicht verfahren werden.

	
	% % % % % % % % % % % % % % %
	
	\section{Infos für Experten}
	
	\subsection{Experteninfos - Makerware} \label{expinfos}
	\subsubsection{Preslicing}
	Preslicing ist beim Druck von SD-Karte nicht notwendig.
	
	Wer schon einmal eine komplexere STL-Datei gesliced hat weiß, dass das dauern kann. Und manchmal gehen Drucke schief,
	oft schon gleich am Anfang. Um jetzt nicht jedes Mal komplett von Null zu beginnen, kann man eine gcode-Dateien erstellen
	und laden.
	\begin{itemize}
		\item zum erstellen bei ''Make`` ''Export to a File`` wählen, nach den Einstellungen auf ''Make It!'' und Datei als \texttt{.gcode} abspeichern
		\item zum Laden auf ''File`` \ra ''Make from File...`` oder Ctrl+Alt+P drücken und entsprechende \texttt{.gcode-Datei} auswählen (Achtung keine Preview!)
	\end{itemize}
	
	\subsection{Kurze Einführung in die wichtigsten Fachbegriffe bei STL:}
	
	\begin{itemize}
		\item Einheit \fachbegriff{unit}: Länge, die der Zahl „1“ entsprechen soll (bei uns
		1 Millimeter)
		\item Punkt \fachbegriff{vertex}: Stelle im Raum
		\item Kante \fachbegriff{edge}: Verbindungslinie zwischen zwei Punkten
		\item Fläche \fachbegriff{face}: Dreieck, das zwischen drei Punkten bzw. zwei
		benachbarten Kanten aufgespannt wird. Im verwendeten STL-Dateiformat
		gibt es nur Dreiecksflächen. Krümmungen werden aus vielen
		Dreiecksflächen angenähert.
		\item Polygonnetz \fachbegriff{mesh}: Gesamtheit aller Flächen, die einen Körper
		ergibt. Alles innerhalb des Netzes soll mit Kunststoff gefüllt werden,
		alles außerhalb ist Luft.
	\end{itemize}
	
	\subsection{Einschränkungen bei Blender und manchen anderen Programmen} \label{lowlevel-einschraenkungen}
	
	Das 3D-Modell muss gewisse Einschränkungen erfüllen, damit die Drucksoftware es versteht. Da man mit Tools wie Blender auch unsinnige Dinge bauen kann (Modelle mit Löchern, unlogische Dinge bei denen innen und außen nicht eindeutig ist), muss auf folgendes geachtet werden.
	\begin{itemize}
		\item Wasserdicht: Der Körper muss rundum geschlossen sein, er darf keine
		Löcher aufweisen,  die Hülle muss ein zusammenhängendes Netz sein.
		\item Polygonnetz \fachbegriff{mesh} schneidet sich nicht selbst: Verschiedene
		Körper dürfen sich nicht überlappen. Sie müssen stattdessen zu einem
		Körper vereinigt werden. In Blender geht dies zum Beispiel mit dem
		Boolean Modifier.
		\item Mannigfaltig \fachbegriff{manifold}: Dieser Begriff ist schwierig zu
		beschreiben. Vereinfacht gesagt dürfen zu jeder Kante nur zwei Flächen
		gehören. Es dürfen also zum Beispiel keine Flächen innerhalb eines
		Körpers existieren.
	\end{itemize}
	
	% % % % % % % % % % % % % %
	
	\section{Infos für Betreuer}
	
	\subsection{Filamentwechsel}\label{filamentwechsel}
	
	\subsubsection{Ultimaker}
	\begin{itemize}
		\item Extruder muss heiß sein: ''Vorbereiten'' \ra ''Vorwärmen PLA/ABS''
		\item ''Vorbereiten'' \ra ''Filament wechseln''
		\item Warten bis komplett herausgefahren
		\item neues Filament einfädeln und mit etwas Druck dagegen halten
		\item Mit Knopfdruck bestätigen
		\item Bis zum extrudieren warten
		\item evtl. auf ''Extrude mehr''  drücken
		\item ''Drucke weiter'' drücken \ra Fertig!
	\end{itemize}
	Alternativ:
	\begin{itemize}
		\item Drucker vorheizen
		\item ''Prepare``\ra ''Move Axis``\ra ''1 mm`` \ra ''Extruder``
		\item Filament aus der Düse zurückfahren
		\item Extruder öffnen
		\item Filament herausziehen
		\item neues Filament einfädeln
		\item ''Extrude``bis Neues Filament extrudiert wird
	\end{itemize}
	
	\subsection{Pflege \& Wartung}
	
	\begin{itemize}
		\item Achsen müssen regelmäßig geölt werden (1x wöchentlich)
		\item Riemenspannung und Riemenzustand überprüfen (leicht gespannt)\\
		Ultimaker: \url{https://www.youtube.com/watch?v=grHmmmSoOfc}
		\item die Ausrichtung und den Zustand der Buildplate überprüfen
		\item die Düse außen putzen und auf Verstopfung untersuchen
		\item lose herumhängende Kabel befestigen
		\item Extruder-Coldend zerlegen und putzen (Filament-Abrieb, Förderwalze zugesetzt)
		\item Extruder-Hotend begutachten und nur wenn unbedingt nötig zerlegen (Temperatursensorkabel ist sehr empfindlich)
	\end{itemize}
	
	\subsection{Firmwareupdates}
	
	\begin{itemize}
		\item den Drucker über USB mit dem Rechner verbinden
		\item PreForm zum Updaten verwenden
	\end{itemize}
	
\end{document}
